The main aim of the experiment discussed hereinafter is the measurement of the lifetime of cosmic-ray muons using plastic scintillators.\\

Before performing any measure, an overview of $\mu^\pm$ origin is given in Chapter \ref{intro}, where some useful data \cite{PDG} are illustrated as well (e.g. muons mean energy at sea level $\sim\SI{4}{GeV}$, the $\cos^2\theta$ angular distribution as a function of the zenith angle, etc).

$\SI{4}{GeV}$ muons are minimum ionizing particles (MIPs), which lose an average energy equal to $\sim \SI{2}{MeV}\,\si{g}^{-1}\,\si{cm}^{2}$ in matter.

Concerning a free decay, the well known quantum field theory prediction for $\mu$ lifetime is $\tau_{\mu}\simeq \SI{2.2}{\micro\second}$. As a matter of fact, this experiment is not carried out in the vacuum, thus we deal with a bound decay (instead of a free one), moreover, further interesting features are given by the expected different behaviors of positive/negative muons due to interactions with matter.\\

According to literature \cite{PDG}, the flux of muons with momentum above $\SI{1}{GeV/c}$ is about $\SI{70}{m}^{-2}\,\si{s}^{-1}\,\si{sr}^{-1}$, but this value refers to a single detector, in fact while employing more than one in coincidence-mode (forming a \emph{telescope}) a flux decrease is expected due to geometrical effects that have been analytically derived (thanks to some assumptions) by J. D. Sullivan \cite{Sullivan} and G. R. Thomas \cite{Thomas}. A Monte Carlo procedure has been developed in order to understand to what extent the assumptions are reasonable.\\

The experimental setup consists of detectors and electronics modules, that have been fully characterized, in order to identify possible systematic uncertainty sources then extract suitable corrections to be employed during physics measures afterward.

Scintillators characterization led to the choice of optimized working conditions, where all detectors provide efficiencies above $90\,\%$ in a state where most of the $\gamma$-background is rejected. The study is completed by uniformity measurements preceded with adequate Monte Carlo simulations where a geometrical weight is evaluated so as to cure the underestimate of efficiency.\\

The last Chapter of this work is dedicated to the $\mu$ lifetime measurements: the Data Acquisition (DAQ) system is explained and some preliminary studies have been carried out. Due to the different $\mu^+$ and $\mu^-$ lifetimes \cite{lifetime}, a Monte Carlo procedure which takes into account the correct muons charge ratio \cite{charge} has been developed, in order to estimate an expected $\tau$.

Sources of systematic errors have been investigated, giving an explanation to the bad behavior observed in decays where the $e^\pm$ are emitted towards the upper detector. However, the collected spectrum referred to down decays has shown a good behavior and it has been fitted both with a single exponential model, giving an estimated $\tau = 2.138\pm \SI{0.033}{\micro\second}$ (compatible with the MC), and with a double exponential model, providing $\tau^- = \left[2.0628 \pm 0.0595\,(stat) \pm 0.0003\,(syst)\right]\,\si{\micro\second}$ and $\tau^+ = \left[2.2251 \pm 0.0759\,(stat) \pm 0.0002\,(syst)\right]\,\si{\micro\second}$, both compatible with the nominal lifetimes in carbon \cite{lifetime}.

