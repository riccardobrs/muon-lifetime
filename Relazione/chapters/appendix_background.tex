\label{Background}

To evaluate the expected background during the experiment there are two ways. The first one is based on the Eq. \eqref{eq:g_Rozza} and the second one on the coincidence-windows of the trigger system.

In the former, starting from \eqref{eq:g_Rozza_adj} which is equivalent to \eqref{eq:g_Rozza} with some adjustments, it is possible to express the muon component $x$ of the incident radiation.
\begin{equation}\label{eq:g_Rozza_adj}
	\frac{R_{double}}{R_{single}} = \frac{\left(x\cdot G_t^\mu + \left(1 - x \right)\cdot G_t^\gamma \right)\varepsilon_3}{x\cdot G_s^\mu + \left(1 - x \right)\cdot G_s^\gamma}
\end{equation}
In this formula $\left( 1 - x\right) $ represents the background fraction, $G_s$ and $G_t$ are the geometric factor for one detectors and for the particle telescope respectively; the apices $\mu$ and $\gamma$ refer to the different angular distribution of the incoming particles, in particular the former refers to a $\cos^2\theta$-distribution while the latter to an isotropic one; $\varepsilon_3$ is the efficiency of the lower scintillator, $R_{double}$ and $R_{single}$ are the coincidence and single detector counting rate. Therefore from Eq. \eqref{eq:g_Rozza_adj} we obtain the background component as reported in Eq. \eqref{eq:back}.
\begin{equation}\label{eq:back}
	1 - x = 1 - \frac{R_{single}\varepsilon_3\cdot G_t^\gamma - R_{double}\cdot G_s^\gamma}{G_s^\mu - G_s^\gamma - R_{single}\varepsilon_3\left( G_t^\mu - G_t^\gamma\right)}
\end{equation}
Unfortunately, we were not able to proceed in this way due to an insufficient amount of data caused by the forced interruption of the laboratory activities. \\

\indent The latter consists in making predictions about the expected uniform background due to random coincidences during lifetime measurements. %signals satisfying the trigger requirements. 
The probability of having an \emph{uncorrelated stop} from a start event within the width of the coincidence gate $\Delta T  = \si{11}{\micro s}$, is
\begin{equation}
	P = (1 - e^{ - R_1 \Delta T}) + (1 - e^{ - R_3 \Delta T})
\end{equation}
where $R_1$ and $R_3$ are single-detector counting rates in the outer scintillators. Since in our case $R_1 \simeq R_3 \simeq \SI{40}{Hz}$ we have $R_i \Delta T \ll 1$, hence we can approximate as
\begin{equation}\label{p}
	P \simeq (R_1 + R_3 ) \cdot \Delta T
\end{equation}
In each bin of width $\dd T \ll \Delta T$ the probability is
\begin{equation}
	P_{bin} \simeq \frac{\dd P}{\dd t} \dd T = (R_1 + R_3 ) \cdot \dd T = const
\end{equation}
Therefore, the uniform background $B$ can be estimated as
\begin{equation}
	B = R_{bkg} \cdot T = P \cdot R_{start} \cdot T 
\end{equation}
and substituting \eqref{p} we obtain
\begin{equation}
	B = (R_1 + R_3 ) \cdot \Delta T \cdot R_{start} \cdot T 
\end{equation}
where $T$ is the run acquisition time and $R_{start}$ is the rate of START events.